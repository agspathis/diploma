\section{Μελλοντικές επεκτάσεις}

\paragraph{} Σκοπός της παρούσας διπλωματικής εργασίας ήταν η ανάπτυξη ενός προσομοιωτή
τσουνάμι ο οποίος παρέχει άμεση πληροφορία σχετικά με την κατανομή της μετάδοσης της ορμής
από το κύμα στην ακτογραμμή. Η εφαρμογή που αναπτύχθηκε στα πλαίσια της εργασίας προσφέρει
αυτήν την πληροφορία σε συνδυασμό με πλήθος άλλων, σε μορφή προορισμένη για εύκολη ανάλυση
και επεξεργασία καθώς και υψηλού επιπέδου οπτικοποιήσεις. Ωστόσο, παρά την πρώτη αυτή
επιτυχία υπάρχουν πολλά περιθώρια επέκτασης τόσο ποιοτικής όσο και ποσοτικής φύσης.

\paragraph{} Σε τεχνικό επίπεδο η εφαρμογή επιτυγχάνει προσομοίωση σημαντικού αλλά όχι
μεγάλου αριθμού σωματιδίων σε σύγκριση με τα σύγχρονα δεδομένα. Το όριο, που βρίσκεται
περίπου στα 85 χιλιάδες σωματίδια δεν κρίνεται ικανοποιητικό, όταν σε τελευταίας
τεχνολογίας \eng{GPU} προσομοιώνονται σε πραγματικό χρόνο υπερδιπλάσια σωματίδια σε
αντίστοιχες προσομοιώσεις. Σημαντικό μέρος του υπολογιστικού φόρτου σχετίζεται με την
συλλογή και εξαγωγή των δεδομένων κατά τη διάρκεια της προσομοίωσης, ωστόσο ο κυριότερος
περιοριστικός παράγοντας (\eng{bottleneck}) είναι η χρήση της \eng{Bullet} σε αυτή. Η
μηχανή αυτή έχει σχεδιαστεί για το χειρισμό πολύπλοκων μηχανικών αλληλεπιδράσεων και ως εκ
τούτου υποστηρίζει πολλές λειτουργίες οι οποίες είναι περιττές σε προσομοιώσεις υψηλής
κανονικότητας και έκτασης, παρόμοιες με αυτές που συζητήθηκαν στην παρούσα εργασία. Ωστόσο
πολλά από τα στοιχεία και μεγέθη που είναι αναγκαία για την υλοποίηση αυτών των
αλληλεπιδράσεων καταγράφονται ούτως η άλλως, με αποτέλεσμα τη συνολική επιβάρυνση της
απόδοσης της εφαρμογής. Η μοναδική ουσιαστική συνεισφορά της \eng{Bullet} στην εφαρμογή
είναι η επίλυση των συστημάτων γεωμετρικών περιορισμών που προκύπτουν ανάμεσα στα
σωματίδια (μέσω του \eng{SIS}), με σκοπό την αύξηση του ελάχιστου απαιτούμενου χρονικού
βήματος και την ευστάθεια της προσομοίωσης υπό το καθεστώς υποδειγματοληψίας στα
εκτεταμένα όρια που εκ φύσεως δημιουργούνται σε ανάλογες προσομοιώσεις ελεύθερης ροής.

\paragraph{} Στον αντίποδα της \eng{Bullet} βρίσκεται το \eng{LP grid} η δομή που
δημιουργήθηκε ειδικά για την οργάνωση των σωματιδίων και την αποθήκευση στοιχείων του
ρευστού με σκοπό τη γρήγορη ανίχνευση αλληλεπιδράσεων και εκτίμηση ποσοτήτων σύμφωνα με
την \eng{SPH}. Η χρήση της \eng{Bullet} αποτέλεσε ανασταλτικό παράγοντα και για την
απόδοση του \eng{LP grid} στην παρούσα υλοποίηση (παράγραφος \ref{ssec:bullet}). Αν και
σαν μηχανή φυσικής παρέχει εκτεταμένη υποδομή, διαχειρίζεται εσωτερικά την αποθήκευση των
σωματιδίων μετά τη δημιουργία τους, επιστρέφοντας στο χρήστη έναν δείκτη (ο οποίος
αποθηκεύεται αντι του σωματιδίου στο διάνυσμα \ttt{particles}). Ένα ακόμη μειονέκτημα της
τρέχουσας σταθερής έκδοσης της \eng{Bullet} (2.83) είναι η σειριακή (\eng{single-thread})
εκτέλεση κώδικα, ο οποίος βάσει ανάλυσης (\eng{profiling}) αντιστοιχεί στο 75\% περίπου
του συνολικού χρόνου εκτέλεσης του προγράμματος (το υπόλοιπο 25\% αντιστοιχεί στον
παράλληλο κώδικα του \eng{SPH} όπως αναφέρθηκε στην παράγραφο \ref{sssec:simulation}). Η
επόμενη κύρια έκδοση της \eng{Bullet} (3.\eng{x}) θα περιλαμβάνει πλήρη υποστήριξη για
φυσική στερεών σωμάτων (\eng{rigid body pipeline}) σε \eng{GPU}, η οποία στην παρούσα φάση
είναι ακόμα δοκιμαστική. Ωστόσο όπως αναφέρθηκε και προηγουμένως, η παρούσα εφαρμογή
χρησιμοποιεί μικρό μέρος των δυνατοτήτων της \eng{Bullet} και λόγω αυτού είναι πιθανό μια
εξειδικευμένη στις απαιτήσεις της μεθόδου υλοποίηση γεωμετρικών περιορισμών σε στενή
συνεργασία με ενα προσαρμοσμένο στη \eng{GPU} \eng{LP grid} να είχε πολύ καλύτερα
αποτελέσματα \cite{goswami2010interactive}. Πέρα από την προσαρμογή της δομής στην
αρχιτεκτονική \eng{SIMD}, θα πρέπει να υλοποιηθεί τουλάχιστον ένα ακριβές σύστημα
καταγραφής των ώσεων του ρευστού προς την ακτογραμμή. Αυτό μπορεί είτε να ακολουθήσει τη
φιλοσοφία των \eng{PBD} όπως στην παρούσα εργασία, με όλα τα πλεονεκτήματα που αυτό
συνεπάγεται, είτε να μοντελοποιηθεί μέσω μίας εκ των διαδεδομένων μεθόδων χειρισμού
οριακών συνθηκών στην \eng{SPH}. Στην περίπτωση που υιοθετηθούν \eng{PBD}, θα πρέπει
επιπρόσθετα να υλοποιηθεί η λειτουργικότητα του \eng{SIS} για την επίλυση των συστημάτων
γεωμετρικών περιορισμών που προκύπτουν. Έχει προταθεί οτι η μέθοδος \eng{PGS (Projected
  Gauss-Seidel)} ταιριάζει πολύ καλύτερα στην παράλληλη αρχιτεκτονική των \eng{GPUs} σε
σύγκριση με την \eng{SIS} \cite{tamis2015}.

\paragraph{} Σε επίπεδο εφαρμογής ο κυριότερος στόχος, δηλαδή η παροχή πλήρους εικόνας της
επίδρασης του κύματος στην ακτογραμμή επετεύχθη σε ικανοποιητικό βαθμό. Ένα από τα
μειονεκτήματα όμως που καθιστά την προσομοίωση μη ρεαλιστική είναι η στατικότητα της
ακτογραμμής, καθώς δεν υπόκειται σε αλλαγές σύμφωνα με τις δυνάμεις που της
ασκούνται. Μεγάλο μέρος των σημιών που προκαλεί ένα τσουνάμι οφείλεται στα αντικείμενα που
δευτερογενώς συμπαρασύρονται από αυτό και προσκρούουν με τη σειρά τους σε άλλα της
ακτογραμμής, ενώ δεδομένης της τεράστιας κλίμακας του φαινομένου αυτά μπορεί να έχουν πολύ
μεγάλη μάζα (όπως αυτοκίνητα, δέντρα, κλπ). Εντούτοις, αφενός η ορμή που συσσωρεύουν
προέρχεται από το κύμα οπότε λαμβάνεται υπόψη σε κάθε περίπτωση, αφετέρου γνωρίζοντας την
ταχύτητα του νερού κοντά σε ένα κτίριο, είναι σχετικά εύκολο να προβλεφθεί η ζημιά που θα
προκληθεί από την πρόσκρουση ενός αντικειμένου δεδομένης μάζας κινούμενου με την ίδια
ταχύτητα. Οι παρατηρήσεις αυτές οδηγούν σε σκέψη για ανάλογες επεκτάσεις στην εφαρμογή της
προσομοίωσης για την πρόβλεψη ζημιών από συμπαρασυρόμενα αντικείμενα, είτε ακόμη καλύτερα
τη προσθήκη δυναμικού χαρακτήρα στην ακτογραμμή. Ένα άλλο επίσης σημαντικό τμήμα του
τσουνάμι το οποίο δεν έχει προσομοιωθεί είναι η αποτράβηξη του νερού από την ακτογραμμή, η
οποία όμως συνήθως δεν προκαλεί σημαντικές επιπρόσθετες ζημιές. Στο πλαίσιο αυτό χρήσιμη
θα ήταν η σύνδεση των στοιχείων του γενεσιουργού συμβάντος με τα χαρακτηριστικά του
κύματος στην ακτογραμμή για την προσαρμογή των αρχικών συνθηκών του ρευστού στα πλαίσια
της προσομοίωσης. Μολονότι ο εισβάλλων όγκος νερού αποτελεί μια αρκετά καλή προσέγγιση για
εκτίμηση των επιπτώσεων ενός τσουνάμι (όπως αναλύθηκε στην παράγραφο
\ref{ssec:simulations}, είναι σημαντική η ποσοτική σύνδεση μεταξύ αυτών και του
γενεσιουργού συμβάντος. Συνολικά, οι κυριότερες μελλοντικές δυνατότητες επέκτασης
εστιάζονται σε θέματα υλοποίησης, χωρίς παράλληλα να αποκλείονται προσαρμογές της
θεωρίας/μοντελοποίησης προς πιο εξελιγμένες τεχνικές προσομοίωσης.


%%% Local Variables:
%%% mode: latex
%%% TeX-master: "report"
%%% End:
