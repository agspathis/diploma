\section{Μελλοντικές επεκτάσεις}

\paragraph{} Ένας ανασταλτικός παράγοντας για την απόδοση του \eng{LP grid} στην παρούσα
υλοποίηση ηταν η χρήση της \eng{Bullet} (παράγραφος \ref{ssec:bullet}). Αν και σαν μηχανή
φυσικής παρέχει εκτεταμένη υποδομή, διαχειρίζεται εσωτερικά την αποθήκευση των σωματιδίων
μετά τη δημιουργία τους, επιστρέφοντας στο χρήστη έναν δείκτη (ο οποίος αποθηκεύεται αντι
του σωματιδίου στο διάνυσμα \ttt{particles}). Ένα ακόμη μειονέκτημα της τρέχουσας σταθερής
έκδοσης της \eng{Bullet} (2.83) είναι η σειριακή (\eng{single-thread}) εκτέλεση κώδικα, ο
οποίος βάσει ανάλυσης (\eng{profiling}) αντιστοιχεί στο 75\% περίπου του συνολικού χρόνου
εκτέλεσης του προγράμματος (το υπόλοιπο 25\% αντιστοιχεί στον παράλληλο κώδικα του
\eng{SPH} όπως αναφέρθηκε στην παράγραφο \ref{sssec:simulation}). Η επόμενη κύρια έκδοση
της \eng{Bullet} (3.\eng{x}) θα περιλαμβάνει πλήρη υποστήριξη για φυσική στερεών σωμάτων
(\eng{rigid body pipeline}) σε \eng{GPU}, η οποία στην παρούσα φάση είναι ακόμα
δοκιμαστική. Ωστόσο η παρούσα εφαρμογή χρησιμοποιεί μικρό μέρος των δυνατοτήτων της
\eng{Bullet} και λόγω αυτού είναι πιθανό μια εξειδικευμένη στις απαιτήσεις της εφαρμογής
υλοποίηση γεωμετρικών περιορισμών/μηχανικής σε στενή συνεργασία με ενα προσαρμοσμένο στη
\eng{GPU} \eng{LP grid} να είχε πολύ καλύτερα αποτελέσματα.

\paragraph{} Σε επίπεδο προσομοίωσης, αν και προσφέρεται πλήρης εικόνα της επίδρασης του
κύματος στην ακτογραμμή, η στατικότητά της καταγράφεται ως μειονέκτημα. Είναι γεγονός οτι
τα δευτερογενώς συμπαρασυρόμενα αντικείμενα αποτελούν παράγοντα ζημιών, δεδομένου οτι
εξαιτίας της κλίμακας των κυμάτων, μπορεί να έχουν πολύ μεγάλη μάζα
(λ.χ. αυτοκίνητα). Εντούτοις, αφενός η ορμή που συσσωρεύουν προέρχεται από το κύμα οπότε
λαμβάνεται υπόψη σε κάθε περίπτωση, αφετέρου γνωρίζοντας την ταχύτητα του νερού κοντά σε
ένα κτίριο, είναι σχετικά εύκολο να προβλεφθεί η ζημιά που θα προκληθεί από την πρόσκρουση
ενός αντικειμένου δεδομένης μάζας κινούμενου με την ταχύτητα αυτή. Μία ακόμη περίπτωση που
δεν έχει προσομοιωθεί είναι η αποτράβηξη του νερού από την ακτογραμμή, η οποία όμως
συνήθως δεν προκαλεί σημαντικές επιπρόσθετες ζημιές. Συνολικά, οι κυριότερες μελλοντικές
δυνατότητες επέκτασης εστιάζονται σε θέματα υλοποίησης, χωρίς παράλληλα να αποκλείονται
προσαρμογές της θεωρίας/μοντελοποίησης προς πιο εξελιγμένες τεχνικές προσομοίωσης.


%%% Local Variables:
%%% mode: latex
%%% TeX-master: "report"
%%% End:
